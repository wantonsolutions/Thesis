\documentclass[12pt]{ucsddissertation}
% mathptmx is a Times Roman look-alike (don't use the times package)
% It isn't clear if Times is required. The OGS manual lists several
% "standard fonts" but never says they need to be used.
\usepackage{mathptmx}
\usepackage[NoDate]{currvita}
\usepackage{array}
\usepackage{tabularx}
\usepackage{booktabs}
\usepackage{ragged2e}
\usepackage{microtype}
\usepackage[breaklinks=true,pdfborder={0 0 0}]{hyperref}
\usepackage{graphicx}
\usepackage{subcaption}
\usepackage{amsmath}
\usepackage{enumitem}
\AtBeginDocument{%
	\settowidth\cvlabelwidth{\cvlabelfont 0000--0000}%
}

%commands

\newcommand{\sword}{SwordBox}
\newcommand{\thesistitle}{Disaggregated Data Structures: Sharing and contention with RDMA and Programmable Networks}

% OGS recommends increasing the margins slightly.
\increasemargins{.1in}

% These are just for testing/examples, delete them
\usepackage{trace}
%\usepackage{showframe} % This package was just to see page margins
\usepackage[english]{babel}
\usepackage{blindtext}
\usepackage{ifthen}
\usepackage[normalem]{ulem} % for \sout
\usepackage{xcolor}
\usepackage{amssymb}

\newcommand{\ra}{$\rightarrow$}
\newboolean{showedits}
\setboolean{showedits}{true} % toggle to show or hide edits
\ifthenelse{\boolean{showedits}}
{
	\newcommand{\ugh}[1]{\textcolor{red}{\uwave{#1}}} % please rephrase
	\newcommand{\ins}[1]{\textcolor{blue}{\uline{#1}}} % please insert
	\newcommand{\del}[1]{\textcolor{red}{\sout{#1}}} % please delete
	\newcommand{\chg}[2]{\textcolor{red}{\sout{#1}}{\ra}\textcolor{blue}{\uline{#2}}} % please change
}{
	\newcommand{\ugh}[1]{#1} % please rephrase
	\newcommand{\ins}[1]{#1} % please insert
	\newcommand{\del}[1]{} % please delete
	\newcommand{\chg}[2]{#2}
}

\newboolean{showcomments}
\setboolean{showcomments}{true}
% \setboolean{showcomments}{false}
\newcommand{\id}[1]{$-$Id: scgPaper.tex 32478 2010-04-29 09:11:32Z oscar $-$}
\newcommand{\yellowbox}[1]{\fcolorbox{gray}{yellow}{\bfseries\sffamily\scriptsize#1}}
\newcommand{\triangles}[1]{{\sf\small$\blacktriangleright$\textit{#1}$\blacktriangleleft$}}
\ifthenelse{\boolean{showcomments}}
%{\newcommand{\nb}[2]{{\yellowbox{#1}\triangles{#2}}}
{\newcommand{\nbc}[3]{
 {\colorbox{#3}{\bfseries\sffamily\scriptsize\textcolor{white}{#1}}}
 {\textcolor{#3}{\sf\small$\blacktriangleright$\textit{#2}$\blacktriangleleft$}}}
 \newcommand{\version}{\emph{\scriptsize\id}}}
{\newcommand{\nbc}[3]{}
 \renewcommand{\ugh}[1]{#1} % please rephrase
 \renewcommand{\ins}[1]{#1} % please insert
 \renewcommand{\del}[1]{} % please delete
 \renewcommand{\chg}[2]{#2} % please change
 \newcommand{\version}{}}
\newcommand{\nb}[2]{\nbc{#1}{#2}{orange}}

\definecolor{ibcolor}{rgb}{0.4,0.6,0.2}
\newcommand\iv[1]{\nbc{IB}{#1}{ibcolor}}
\usepackage{wasysym}
\newcommand\yesml[1]{\nbc{ML {\textcolor{yellow}\sun}}{#1}{mircolor}}

\definecolor{sgcolor}{rgb}{0.2,0.0,0.5}
\newcommand\sg[1]{\nbc{SG}{#1}{sgcolor}}

\definecolor{samcolor}{rgb}{0.2,0.4,0.2}
\newcommand\sam[1]{\nbc{SC}{#1}{samcolor}}

\definecolor{hccolor}{rgb}{0.21,0.54,0.84}
\newcommand\hc[1]{\nbc{HC}{#1}{hccolor}}

\definecolor{ideacolor}{rgb}{1.0,0,0.5}
\newcommand\idea[1]{\nbc{IDEA}{#1}{ideacolor}}


\definecolor{abstractcolor}{rgb}{0.0,0.5,1.0}
\newcommand\rabstract[1]{\nbc{ABSTRACT}{#1}{abstractcolor}}

\definecolor{introcolor}{rgb}{0.0,1.0,0.5}
\newcommand\rintro[1]{\nbc{INTRO}{#1}{introcolor}}

\definecolor{papercolor}{rgb}{1.0,1.0,0.0}
\newcommand\rpaper[1]{\nbc{PAPER}{#1}{papercolor}}

\definecolor{multicolor}{rgb}{1.0,0,0}
\newcommand\rmulti[1]{\nbc{MULTI}{#1}{multicolor}}

% Todo Command
\definecolor{todocolor}{rgb}{0.9,0.1,0.1}
\newcommand{\todo}[1]{\nbc{TODO}{#1}{todocolor}}


\overfullrule5pt
% ---

% Required information
\title{\thesistitle}
\author{Stewart Steven Grant}
\degree{Computer Science}{Doctor of Philosopy}
% Each member of the committee should be listed as Professor Foo Bar.
% If Professor is not the correct title for one, then titles should be
% omitted entirely.
\chair{Alex C. Snoeren}
% Your committee members (other than the chairs) must be in alphabetical order
\committee{Amy Ousterhout}
\committee{Yiying Zhang}
\committee{George Papen}
\degreeyear{2024}

% Start the document
\begin{document}
% Begin with frontmatter and so forth
\frontmatter
\maketitle
\makecopyright
\makesignature
% Optional
\begin{dedication}
\setsinglespacing
\raggedright % It would be better to use \RaggedRight from ragged2e
\parindent0pt\parskip\baselineskip

\textit{To my inner circle -- Family, Loved ones and Friends. And to every belayer that caught me. Thanks for the proverbial and literal support.}


% In recognition of reading this manual before beginning to format the
% doctoral dissertation or master's thesis; for following the
% instructions written herein; for consulting with OGS Academic Affairs
% Advisers; and for not relying on other completed manuscripts, this
% manual is dedicated to all graduate students about to complete the
% doctoral dissertation or master's thesis.

% In recognition that this is my one chance to use whichever
% justification, spacing, writing style, text size, and/or textfont that
% I want to while still keeping my headings and margins consistent.
\end{dedication}
% Optional
\begin{epigraph}
\vskip0pt plus.5fil
\setsinglespacing
{\flushright

"If this is the best of possible worlds, what then are the others?"\\
\vskip\baselineskip
-Voltaire \textit{Candide}\par}

\vfil
\begin{center}

\noindent “It must be considered that there is nothing more difficult to carry out, nor more
doubtful of success, nor more dangerous to handle, than to initiate a new order of things.”

\vskip\baselineskip
\hskip0pt plus1fil -Niccolo Machiavelli \textit{The Prince}\hskip0pt plus4fil\null

\end{center}
\vfil
"If every porkchop were perfect we wouldn't have hotdogs"\\

\vskip\baselineskip
-Greg Universe, \textit{Steven Universe}

\vfil
\end{epigraph}

% Next comes the table of contents, list of figures, list of tables,
% etc. If you have code listings, you can use \listoflistings (or
% \lstlistoflistings) to have it be produced here as well. Same with
% \listofalgorithms.
\tableofcontents
\listoffigures
\listoftables

% Preface
\begin{preface}
\todo{Write a preface - take a look at some examples because it seems rather free form}
Almost nothing is said in the manual about the preface. There is no
indication about how it is to be typeset. Given that, one is forced to
simply typeset it and hope it is accepted. It is, however, optional
and may be omitted.
\end{preface}

% Your fancy acks here. Keep in mind you need to ack each paper you
% use. See the examples here. In addition, each chapter ack needs to
% be repeated at the end of the relevant chapter.
\begin{acknowledgements}

I would like to acknowledged my advisor Alex C. Snoeren for his dedication to his craft and guidance
over the past 6 years. No piece of work within this thesis would be possible without your
collaboration. I would also like to thank my committee members Amy Ousterhout, Yiying Zhang, and
George Papen for their feedback and guidance, and Shrikanth Kandula for his mentorship during my
time at MSR.

This thesis has been extraordinarily influenced by Anil Yelam, my closest collaborator. Thank you
for all the time you spent working on our collaborations, and the hours spent discussing and
debating system designs and performance results. I'm forever grateful. To Maxwell Bland, your
research energy is unmatched and without your help we would never have have acquired any SmartNICs.
And Alex (Enze) Liu for his superior knowledge of Python and unmatched focus on research.  Thank you
to all of the members of the Systems and Networking group at UCSD, especially the optical networking
group for your feedback and guidance during the first years of my PhD.

Thank you to Meta for funding my research and providing me with the opportunity to work on resource
disaggregation, Cavium for the generous donation of two SmartNICs, and to ARPAe for funding my first
years of research.

I'd like to thank all of the members of 3140 for their collaboration and friendship over the years.
It's truly the best office, Chez bob volunteers for keeping me fed, and to my friends for the
support. Camille Rubel thanks for having the best climbing schedule in the world, Phillip Arndt for
pushing my limits, and Camille Moore for keeping me on my toes.
\todo{double check I thanked everyone}


\end{acknowledgements}

% Stupid vita goes next
\begin{vita}
\noindent
\begin{cv}{}
\begin{cvlist}{}
\item[2012-2016] Bachelor of Science, Computer Science University of British Columbia
\item[2016-2018] Master of Science, Computer Science University of British Columbia
\item[2018-2024] Doctor of Philosophy, Computer Science University of California, San Diego
\end{cvlist}
\end{cv}

% This puts in the PUBLICATIONS header. Note that it appears inside
% the vita environment. It is optional.
\publications

\noindent Deepak Bansal and Gerald DeGrace and Rishabh Tewari and Michal Zygmunt and James Grantham
and Silvano Gai and Mario Baldi and Krishna Doddapaneni and Arun Selvarajan and Arunkumar Arumugam
and Balakrishnan Raman and Avijit Gupta and Sachin Jain and Deven Jagasia and Evan Langlais and
Pranjal Srivastava and Rishiraj Hazarika and Neeraj Motwani and Soumya Tiwari and Stewart Grant and
Ranveer Chandra and Srikanth Kandula. 2023. Disaggregating Stateful Network Functions. In
proceedings of 20th USENIX Symposium on Networked Systems Design and Implementation (NSDI 23).
Usenix Association, Boston MA, USA, 1469--1487.
https://www.usenix.org/conference/nsdi23/presentation/bansal \\

\noindent Stewart Grant, Anil Yelam, Maxwell Bland, and Alex C. Snoeren. 2020. SmartNIC Performance
Isolation with FairNIC: Programmable Networking for the Cloud. In Proceedings of the Annual
conference of the ACM Special Interest Group on Data Communication on the applications,
technologies, architectures, and protocols for computer communication (SIGCOMM '20). Association for
Computing Machinery, New York, NY, USA, 681–693. https://doi.org/10.1145/3387514.3405895 \\

\noindent Stewart Grant, Hendrik Cech, and Ivan Beschastnikh. 2018. Inferring and asserting
distributed system invariants. In Proceedings of the 40th International Conference on Software
Engineering (ICSE '18). Association for Computing Machinery, New York, NY, USA, 1149–1159.
https://doi.org/10.1145/3180155.3180199 \\


% This puts in the FIELDS OF STUDY. Also inside vita and also
% optional.
% \fieldsofstudy
% \noindent Major Field: Computer Science
\end{vita}

% Put your maximum 350 word abstract here.
\begin{dissertationabstract} 
%%	
Resource disaggregation proposes a next-generation architecture for data center resources. System
components like compute, memory, storage, and accelerators are separated from one another by a fast
network and composed dynamically into virtual servers when required. This paradigm promises
dramatically improved resource utilization, scalability, and flexibility, but introduces dramatic
challenges in terms of performance and fault tolerance. Memory is among the most difficult resources
to disaggregate. CPUs currently expect DRAM to ultra low latency, high bandwidth, and to share it's
failure domain. In particular increased latency from network round trips dramatically shifts the
performance of existing shared data structures designed for local DRAM.
%%
In this thesis I explore the challenges of sharing disaggregated memory with hundreds of CPU cores.
First in {\sword} I present a system which utilizes a centralized programmable switch to cache data
structure state and dramatically improve key-value workload performance. Second I present a new
key-value store RCuckoo which is designed to leverage RDMA and reduce round trips when accessed by
CPU's over a network. Finally I present Black Box Disaggregation, a system which enables arbitrary
data structures to be easily ported to the disaggregated setting.

\end{dissertationabstract}

% This is where the main body of your dissertation goes!
\mainmatter

% Optional Introduction
\begin{dissertationintroduction}
	\todo{Big todo write the diseration introduction}
\end{dissertationintroduction}

\chapter{Background}

Calls for distributed memory systems have been made for decades, and the groundwork of
disaggregation relies heavily on the work of the past~\cite{treadmarks,gms}. However, disaggregation
differs from prior distributed systems due to it's differences in scale, and system assumptions.
It's primary difference is the assumption that each resource be stand alone, and paired with
miniscule or no compute power, save for resources like CPUs an GPUs which are inherently compute.
This assumption transforms the technologies and requirements for disaggregated systems. Critically
Remote Direct Memory Access (RDMA) (or similar CPU bypass technology) is essential in constructing a
passive storage system. Simultaneously in-network line rate processing of user defined functions is
now commonly available in a variety of network devices. The intersection of passive storage,
massively concurrent compute, and programmable networks set a unique landscape for the design of
disaggregated systems.

\section{Disaggregation}

As with many paradigm shifts within the systems community the idea of disaggregation is born from
shifting hardware trends. Per core access to memory has been decreasing for decades. CPU core counts
have consistently increased for over a decade while memory speeds and capacity have improved
comparatively slowly~\todo{cite memory chart}. The result is that CPU cores have less access to
memory now than 10 years ago. Memory is now an increasingly scarce resource and data center
operators are exploring new ways to achieve better memory utilization. Disaggregation is one such
option. Monolithic servers have a fixed amount of RAM per machine which, at a data center level leads
to a skewed distributed of memory utilization. Some servers are starving for memory while others
have gigabytes to spare. This spare (stranded) memory is the target of disaggregation, it aims to
give each server access to a shared pool of RAM. In general a disaggregated resource, be it memory,
FPGA's, or the network itself, can be provisioned for the peak-of-sums, rather than monolithic
resources which are provisioned for the sum-of-peaks~\cite{clio,supernic,dsnf}.

\section{RDMA}

\section{Programable Networks}

\section{Disaggregated Data Structures}



\chapter{Swordbox: Accelerated Sharing of Disaggregated Memory}

\input{swordbox/intro}
\input{swordbox/backv2}
\input{swordbox/tmp_sigcomm}
\input{swordbox/implementation}
\input{swordbox/eval}
\input{swordbox/limitations}
\input{swordbox/conclusion}

\section{Overview}

\todo{Example for the bibliograph, here is a cite to clover~\cite{clover}}

\todo{Revise Swordbox and then paste it into this section}

\chapter{Disaggregated Cuckoo Hashing}
\input{rcuckoo/intro}
\input{rcuckoo/background}
\input{rcuckoo/body}
%\input{problems}
\input{rcuckoo/design}
\input{rcuckoo/evaluation}
%\input{limitation}
\input{rcuckoo/conclusion}

\chapter{Black Box Disaggregation}

\section{Overview}
\todo{finish up you writeup for a fall submission of BBD and paste it her}

\chapter{Conclusion}

\chapter{ Example Figures and such for formatting reference}
This demonstrates how OGS wants figures and tables formatted. For
figures, the caption goes below the figure and ``Figure'' is in bold.
See Figure~\ref{fig:zen}. Tables are formatted with the caption above
the table. See Table~\ref{tab:bad}.

Of course, Table~\ref{tab:bad} looks horrible. It should probably be
formatted like Table~\ref{tab:good} instead.

For facing caption pages, see Table~\ref{tab:facing}. Of course,
facing caption pages are vaguely ridiculous and my implementation of
them in the class file is by far the most brittle part of the
implementation. It's entirely possible that something has changed and
these don't work at all. I implemented it merely for the challenge.

\begin{figure}
\centering
\fbox{\parbox{.9\linewidth}{%
	\noindent
	{\Huge PHD ZEN}\par
	\vskip.5in
	\centerline{comic here}
	\vskip.5in
}}
\caption[``Ph.D. Zen'']{Comic entitled ``Ph.D. Zen'' by Jorge Cham, 2005. Copyright
has not been obtained and so it isn't displayed.}
\label{fig:zen}
\end{figure}

\begin{table}
\centering
\caption[Electronic Dissertation Submission Rates]{Electronic
Dissertation Submission Rates at UCSD, Fall 2005 and Winter 2006.
(First two quarters that the program was available to all Ph.D.
candidates not in a Joint Doctoral Program with SDSU.)}
\label{tab:bad}
\begin{tabular}{|*{5}{>{\centering\arraybackslash}m{.15\linewidth}|}}
\hline
&Ph.D.s awarded (Including Joint degrees) & Electronic submission of
Dissertation & Paper Submission of Dissertation & Percentage of
Electronic Submission\\
\hline
Fall\par 2005 & 84 & 37 & 47 & 44.05\%\\
\hline
Winter\par 2006 & 64 & 42 & 22 & 65.63\%\\
\hline
\end{tabular}
\end{table}

\begin{table}
\centering
\caption[Electronic Dissertation Submission Rates]{Electronic
Dissertation Submission Rates at UCSD, Fall 2005 and Winter 2006.
(First two quarters that the program was available to all Ph.D.
candidates not in a Joint Doctoral Program with SDSU.)}
\label{tab:good}
\renewcommand\tabularxcolumn[1]{>{\RaggedRight\arraybackslash}p{#1}}
\begin{tabularx}{.9\linewidth}{lcccc}
\toprule
&\multicolumn{1}{X}{Ph.D.s awarded (Including Joint degrees)}
&\multicolumn{1}{X}{Electronic submission of Dissertation}
&\multicolumn{1}{X}{Paper Submission of Dissertation}
&\multicolumn{1}{X}{Percentage of Electronic Submission}\\
\midrule
Fall 2005 & 84 & 37 & 47 & 44.05\%\\
Winter 2006 & 64 & 42 & 22 & 65.63\%\\
\bottomrule
\end{tabularx}
\end{table}

\begin{facingcaption}{table}
\caption[UCSD Gender Distribution]{University of
California, San Diego Gender Distribution for the Campus Population,
October~2005\\
(http://assp.ucsd.edu/analytical/Campus\%20Population.shtml)\\
\emph{(This is an example of a facing caption page, the next page is
the example of the table/figure/etc.\ that corresponds to this
caption. It is also an example of table/figure that is rotated 90
degrees to fit the page.)}}
\label{tab:facing}
\renewcommand\tabularxcolumn[1]{>{\RaggedLeft\arraybackslash}p{#1}}
\parindent=0pt
\setbox0=\vbox}
& \multicolumn{1}{c}{\textbf{N}} & \multicolumn{1}{c}{\textbf{\%}}
& \multicolumn{1}{c}{\textbf{N}} & \multicolumn{1}{c}{\textbf{\%}}\\
\midrule
Students & 12,987 & 51\% & 12,686 & 49\% & 25,673 & 100\%\\
Employees & 9,943 & 56\% &  7,671 & 44\% & 17,614 & 100\%\\
\addlinespace
\hfill\textbf{Total} & \textbf{22,930} & \textbf{53\%} &
\textbf{20,357} & \textbf{47\%} & \textbf{43,287} & \textbf{100\%}\\
\bottomrule
\end{tabularx}
\singlespacing

\emph{Notes}:
\begin{enumerate}
\item The counts shown below will differ from the official quarterly
Registrar's registration report because 1) data for residents in the
Schools of Medicine and Pharmacy and Pharmaceutical Science are
excluded, and 2) registered, non-matriculated, visiting students are
included.
\item Student workers are excluded from employees; however emeritus
faculty and others on recall status are included.
\end{enumerate}

Campus Planning. Analytical Studies and Space Planning\\
31 January 2006
}
\centerline{\rotatebox{90}{\box0}}
\end{facingcaption}

% This will give us some more text
% \Blinddocument

% Skipping a bunch of chapters
\begin{figure}
\centering
\fbox{\hbox to.8\linewidth{\hss Another figure\hss}}
\caption{Another figure caption}
\end{figure}
\begin{table}
\centering
\caption{Another table caption}
\begin{tabular}{ccc}
\toprule
X&Y&Z\\
\midrule
a&b&c\\
\bottomrule
\end{tabular}
\end{table}
\begin{figure}
\caption{ASDF fig}
\end{figure}
\begin{table}
\caption{ASDF tab}
\end{table}

\appendix
% \Blinddocument
\bibliographystyle{plain} % Or whatever style you want like plainnat

\bibliography{thesis}

% Stuff at the end of the dissertation goes in the back matter
\backmatter

\end{document}
